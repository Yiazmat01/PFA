\section{Organisation}

\begin{frame}
\tableofcontents[currentsection]
\end{frame}
\begin{frame}

\frametitle{Analyse de l'organisation de l'équipe}
Points positifs et négatifs de la coordination au sein du groupe :

\bigskip
  \begin{columns}[t]
  \begin{column}{5cm}
  \begin{exampleblock}{Points positifs}
\begin{itemize}
    \item[+] travail en binôme
      \begin{itemize}
      \item habituel pour le groupe
      \item plus grande efficacité
      \item avancement simultané
      \end{itemize}
    \item[+] familiarisation avec des outils standard
    \item[+] bonnes relations avec les clients
  \end{itemize}
  \end{exampleblock} 
  \end{column}
  
  \begin{column}{5cm}
  \begin{alertblock}{Points négatifs}
\begin{itemize}
    \item[-] communication parfois insuffisante ou inefficace
    \item[-] difficultés pour tenir les délais
      \begin{itemize}
      \item difficultés techniques
      \item calendrier chargé
      \end{itemize}
      
    \end{itemize}
  \end{alertblock}   
  \end{column}
  \end{columns}  

\end{frame}

\begin{frame}
\frametitle{Retour sur les dates importantes}
\begin{itemize}
\item \textbf{23 octobre} : première rencontre avec les clients pour définir les objectifs du projet
\item \textbf{6 décembre} : présentation et recueil des remarques concernant le cahier des charges : début du développement à proprement parler
\item \textbf{7 février} : présentation des fonctionnalités implémentées à cette date, premières impressions
\item \textbf{7 mars} : remise d'une version sur clé dans le but d'être testée par les professeurs
\item \textbf{31 mars} : fin du projet, livrable final
\end{itemize}
\end{frame}