\section{Perspectives d'évolution}

\begin{frame}
  \tableofcontents[currentsection]
\end{frame}
% le quiz n'est pas restreint a de la musique
\begin{frame}{Nouveaux outils}
  Du fait de notre programmation modulaire, il est possible facilement d'ajouter des logiciels outils indépendamment du projet en le rajoutant à \texttt{ToolWidget.cpp}
  Par exemple : \\
  \begin{itemize}
  \item Un accordeur
  \item Un enregistreur simplifié
  \item Un outil ressemblant band-in-the-box
  \item Et à peu près tout logiciel, à la fantaisie du développeur/ du client
  \end{itemize}
  
  Il faut cependant faire attention aux divers problèmes de garantie et à la taille du logiciel pour pouvoir le garder sur la clef.
\end{frame}

\begin{frame}{Le quizz-1}
Le quizz est indépendant de toute notion de musique et est donc réutilisable en dehors de notre logiciel.\\
\begin{block}{Cependant il peut être bon de modifier :}
  \begin{itemize}
  \item la notation, pour prendre en compte la rapidité de réponse
  \item l'affichage du score en cours, en fonction des attentes des élèves
  \item la sécurité de l'administration pour éviter que les élèves trichent
  \end{itemize}
\end{block}
\end{frame}
\begin{frame}{Le quizz-2}
\begin{block}{Autres améliorations possibles : }
  \begin{itemize}
  \item Disposer d'un ensemble de questions/thèmes de base pour le quizz.
  \item Disposer d'une base de données de questions/thèmes sur le net.
  \item Disposer d'un moyen de permettre aux élèves de comparer leurs scores sur les différents quizz.
  \end{itemize}  
  Une solution pourrait être d'avoir un site web, qui permettrait également de mettre à jour les clefs des élèves.
\end{block}
\end{frame}

\begin{frame}{Le jeu}
  on peut mettre un jeu qui déchire sa grand-mere en Shit Fuck My Life
\end{frame}

% partage de la base de donnees du quizz
%revoir la notation du quizz
%