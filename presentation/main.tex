\documentclass[french]{beamer}

\usepackage[utf8]{inputenc}
\usepackage[T1]{fontenc}
\usepackage{lmodern}
\usepackage{amsmath, amssymb}

\usepackage{babel}
\setbeamertemplate{itemize item}[ball]
\setbeamertemplate{itemize subitem}[triangle]
\setbeamertemplate{itemize subsubitem}[circle]

%CHOIX DU THEME et/ou DE SA COULEUR
% => essayer différents thèmes 
% voir http://mcclinews.free.fr/latex/beamergalerie/completsgalerie.html
\usetheme{PaloAlto}
%\usetheme{Madrid}
%\usetheme{AnnArbor}
%\usetheme{Copenhagen}

% voir http://mcclinews.free.fr/latex/beamergalerie/colorgalerie.html
%\usecolortheme{crane}
%\usecolortheme{seahorse}
%\usecolortheme{albatross}
\usecolortheme{dolphin}
%\useoutertheme[left]{sidebar}


%Pour le TITLEPAGE
\title{Exemple de Beamer}
\subtitle{Initiation master~1}
\author[Nom (court)]{Nom (long) de l'auteur}
\date{Avril 2014}
\institute[UT3 -- FSI]{Université Toulouse~3 -- Faculté des sciences et ingénierie}


\begin{document}

\begin{frame}
	\titlepage
\end{frame}

\begin{frame}
	Un environnement \texttt{frame} pour chaque \emph{diapositive}.
	\visible<2>{Chaque diapo pouvant contenir plusieurs \emph{couches}.}
\end{frame}


\begin{frame}{On peut mettre un titre : Sommaire}
	\tableofcontents
\end{frame}

\section{Exemple}
\begin{frame}{La section 1 commence}
	blabla
\end{frame}

\begin{frame}
	Un \textbf<2,3>{texte} en gras. 
	\visible<3>{Un texte visible sur la 3\ieme{} couche}
\end{frame}

\begin{frame}{Titre (facultatif)} 
\framesubtitle{Sous titre (facultatif aussi)}
	\begin{block}{Remarque}
	Un bloc
	\end{block}
	
	\begin{alertblock}{Proposition}
	Un bloc alerte
	\end{alertblock}
	
	\begin{exampleblock}<2>{Exemple}
	Un bloc exemple qui est visible sur la 2\ieme{} couche : $f(x)=2x$.
	\end{exampleblock}
\end{frame}

\begin{frame}{La section 2 commence}
 voir d'autres exemples dans le dossier exemple.
\end{frame}

\section{Présentation}
\begin{frame}
\tableofcontents[currentsection]
\end{frame}
\begin{frame}

  \frametitle{Quelles sont les attentes ?}
  
  \centering \huge{Quelles sont les attentes ?}

\end{frame}

\begin{frame}
  \frametitle{Les attentes fonctionnelles}
  \Large{Plusieurs fonctionnalités sont attendues :}
\bigskip
\bigskip
  \begin{columns}[t]
    \begin{column}{.5\textwidth}
      \centering \large Quizz
      \begin{figure}[!h]

        \begin{center}
          \includegraphics[width = 0.6\textwidth]{presentation/quizz.png}
         \end{center}
      \end{figure}
    \end{column}
    \begin{column}{.5\textwidth}
      \centering \large Jeu(x)
      \begin{figure}[!h]

        \begin{center}
          \includegraphics[width = 0.6\textwidth]{presentation/jeu.png}
        \end{center}
      \end{figure}
      
    \end{column}
    
  \end{columns}

\end{frame}

\begin{frame}
  \frametitle{Les attentes fonctionnelles}

  \centering \Large Plusieurs fonctionnalités sont attendues :
\bigskip
  \begin{columns}[t]
    \begin{column}{0.5\textwidth}

      \centering \large Réglages et administration
      \begin{figure}[!h]
        \bigskip
        \begin{center}
          \includegraphics[width = 0.45\textwidth]{presentation/admin.png}
        \end{center}
      \end{figure}
    \end{column}
    \begin{column}{0.5\textwidth}
      \centering \large Outils musicaux
      \begin{figure}[!h]
        \begin{center}
          \includegraphics[width = 0.3\textwidth]{presentation/outils1.png}
        \end{center}
      \end{figure}
      
      \begin{figure}[!h]
        \begin{center}
          \includegraphics[width = 0.60\textwidth]{presentation/outils2.png}
        \end{center}
      \end{figure}
    \end{column}
    
  \end{columns}
  
\end{frame}


\begin{frame}


  \frametitle{Les autres attentes}
  \large D'autres sont moins fonctionnelles mais nécessaires :
  \bigskip
  \begin{columns}[t]
    \begin{column}{0.5\textwidth}

      \centering \normalsize Logiciel ludique
      \begin{figure}[!h]
        \bigskip
        \begin{center}
          \includegraphics[width = 0.45\textwidth]{presentation/des.png}
        \end{center}
      \end{figure}
    \end{column}
    \begin{column}{0.5\textwidth}
      \centering \normalsize Multi-plateformes
      \begin{figure}[!h]
        \begin{center}
          \includegraphics[width = 0.30\textwidth]{presentation/windows.png}
        \end{center}
      \end{figure}
      
      \begin{figure}[!h]
        \begin{center}
          \includegraphics[width = 0.45\textwidth]{presentation/macos.png}
        \end{center}
      \end{figure}
    \end{column}
    
  \end{columns}
\end{frame}

\begin{frame}


  \frametitle{Les autres attentes}
  \large D'autres sont moins fonctionnelles mais nécessaires :
\bigskip
  \begin{columns}[t]
    \begin{column}{0.5\textwidth}

      \centering \normalsize Prise en main intuitive
      \begin{figure}[!h]
        \bigskip
        \begin{center}
          \includegraphics[width = 0.6\textwidth]{presentation/solution.jpg}
        \end{center}
      \end{figure}
    \end{column}
    \begin{column}{0.5\textwidth}
      \centering \normalsize Système évolutif
      \begin{figure}[!h]
        \bigskip
        \begin{center}
          \includegraphics[width = 0.8\textwidth]{presentation/evolution.png}
        \end{center}
      \end{figure}

    \end{column}
    
  \end{columns}
\end{frame}

\section{Notre solution : la clé USB}
\begin{frame}
\tableofcontents[currentsection]
\end{frame}
\begin{frame}
lorem lipsum
\end{frame}
\begin{frame}

\end{frame}
\section{Organisation}

\begin{frame}
\tableofcontents[currentsection]
\end{frame}
\begin{frame}
lorem lipsum
\end{frame}
\begin{frame}
 
\end{frame}
\section{Conception et développement}

\begin{frame}
\tableofcontents[currentsection]
\end{frame}

\begin{frame}
	\frametitle{Conception base de données}
	Comment stocker correctement les questions/réponses ?
	
	\begin{figure}[!h]
		\centering
			\includegraphics[width = 0.9\textwidth]{conception/bdd.png}
		\label{Schéma base de données} 
		\caption{Schéma base de données}
	\end{figure}
\end{frame}

\begin{frame}
	\frametitle{Conception base de données}
	Discussion du modèle utilisé : dépend des critères recherchés
		
	\bigskip

	\begin{columns}[t]
		\begin{column}{10cm}
			\begin{exampleblock}{Notre architecture}
				\begin{itemize}
					\item[-] est plus complexe au premier abord
					\item[-] nécessite un peu d'algorithmique
				\end{itemize}
			\end{exampleblock} 
		\end{column}
	\end{columns}
	
	\begin{columns}[t]
		\begin{column}{10cm}
			\begin{exampleblock}{Mais au final}
				\begin{itemize}
					\item[+] l'évolution et la maintenance sont facilitées
				\end{itemize}
			\end{exampleblock} 
		\end{column}
	\end{columns}
\end{frame}

\begin{frame}
	\frametitle{Architecture de classes}
	Calquée sur un modèle MVC (Modèle, Vue, Contrôleur)
	
	\begin{figure}[!h]
		\centering
			\includegraphics[width = 0.9\textwidth]{conception/classes.png}
		\label{Schéma des classes} 
		\caption{Schéma des classes}
	\end{figure}
\end{frame}

\begin{frame}
	\frametitle{Développement}
	Le modèle MVC nous a facilité le développement
		
	\begin{columns}[t]
		\begin{column}{10cm}
			\begin{exampleblock}{Avantages de MVC}
				\begin{itemize}
					\item[+] répartition du travail en équipes de deux
					\item[+] chaque groupe est indépendant
					\item[+] chaque module offre des services aux autres
				\end{itemize}
			\end{exampleblock} 
		\end{column}
	\end{columns}
	
	\bigskip
	
	Nécessite une conception rigoureuse en amont afin de pouvoir tout fusionner
	sans problème à la fin!
\end{frame}

\begin{frame}
	\frametitle{Développement}
	Les technologies utilisées sont inter-compatibles.
		
	\begin{columns}[t]
		\begin{column}{10cm}
			\begin{exampleblock}{Technologies utilisées}
				\begin{itemize}
					\item Langage de base : C++ orienté objet
					\item Qt : C++ orienté objet
					\item SFML : C++ orienté objet
				\end{itemize}
			\end{exampleblock} 
		\end{column}
	\end{columns}
	
	\bigskip
	
	=> Code homogène et découpage en classes intuitif
\end{frame}
\begin{frame}
 
\end{frame}
\section{Problèmes rencontrés}

\begin{frame}
\tableofcontents[currentsection]
\end{frame}
\begin{frame}
lorem lipsum
\end{frame}
\begin{frame}
 
\end{frame}
\section{Perspectives d'évolution}

\begin{frame}
\tableofcontents[currentsection]
\end{frame}
\begin{frame}
lorem lipsum
\end{frame}
\begin{frame}
 
\end{frame}
\input{bonus_track.tex}


\end{document}